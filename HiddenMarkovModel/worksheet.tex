\documentclass{article}
\usepackage[margin=1in]{geometry}
\usepackage{float}
\usepackage{xcolor}
\usepackage{amsmath}
\usepackage{makecell}

% special commands for formatting
\newcommand{\fillspace}{\hspace{.95in}}
\newcommand{\inftygrey}{\textcolor{lightgray}{$-\infty$}}
\newcommand{\cellspace}{\Gape[1in]}
\setlength\parindent{0pt}

% name of worksheet
\newcommand{\worksheetname}{Part of Speech Tagging with Hidden Markov Models}


\begin{document}

{\Large\textbf{\worksheetname}}\\\rule{\linewidth}{0.5mm}
\pagenumbering{gobble}

\section{Sequence Probability} You are trying to compute the log probability of the tag sequence ``\texttt{V P V}'' given the sequence of words ``\texttt{ski on snow}''. This would be represented as $$\log(P(\texttt{V P V}|\texttt{ski on snow}))$$

Remember that with the assumptions we will make in our HMM tagger,
\begin{equation*}
\log{(P(t_1...t_n|w_1...w_n))}=\log{ \left( \prod_{i=1}^{n} P(w_i|t_i)P(t_i|t_{i-1})\right)} = \sum_{i=1}^{n} \left( \log{P(w_i|t_i)} + \log{P(t_i|t_{i-1})} \right) 
\end{equation*}

We will state that $P(t_1|t_{0}) = P(t_1|<s>)$, where \texttt{<s>} is a special tag indicating the start of the sequence.

\subsection{Which probabilities?}
Without writing out any numerical probabilities, complete equation below to show the conditional probabilities you must add.\\

\noindent $\log{(P(\texttt{ski}|\texttt{V}))} + \log{(P(\texttt{V}|\texttt{<s>}))} +$

\vspace{1in}
\subsection{Plugging in numbers}

Imagine your HMM is defined by the following initial, transition, and emission \textbf{log} probabilities. The initial probabilities represent $P(t|<\text{s}>)$.

\begin{table}[H]
\centering
\begin{tabular}{l|rrr}
tag & V & N & P \\ \hline
log prob & -3 & -3 & -3
\end{tabular}
\caption{Initial \textbf{log} probabilities}
\end{table}

\begin{minipage}{0.5\textwidth}
\begin{table}[H]
    \centering
    \begin{tabular}{l|rrr}
         & \multicolumn{3}{c}{second tag}\\
         first tag &  V& N & P\\ \hline 
         V& -4 & -2 & -2 \\
         N& -3 & -2 & -1 \\
         P& -5 & -2 & -4\\
    \end{tabular}
    \caption{Transition \textbf{log} probabilities}
\end{table}
\end{minipage}
\begin{minipage}{0.5\textwidth}
\begin{table}[H]
    \centering
    \begin{tabular}{l|rrr} 
         & \multicolumn{3}{c}{word}\\
         tag &  ski&  on& snow\\ \hline 
         V&  -6 & \inftygrey & -5 \\
         N&  -5 & \inftygrey & -3\\ 
         P& \inftygrey & -1 & \inftygrey \\
    \end{tabular}
    \caption{Emission \textbf{log} probabilities}
\end{table}
\end{minipage}

\vspace{.25in}

\noindent Use the probabilities above to compute $$\log(P(\texttt{V P V}|\texttt{ski on snow}))$$

\newpage

\section{Viterbi Algorithm}
\subsection{Completing the $viterbi$ table}
Using the initial, transition, and emission \textbf{log probabilities} from the previous page, fill out the table representing the execution of the Viterbi algorithm. Your final answers in each box should be in the form of log probabilities (for $argmax$, log probabilities are suitable). Keep track of backpointers, either by writing arrows or writing the tag that led to the maximum probability in each box you fill out.

\begin{table}[!ht]
    \centering
    \begin{tabular}{|c|@{\fillspace}c@{\fillspace}|@{\fillspace}c@{\fillspace}|@{\fillspace}c@{\fillspace}|} \hline 
         &   ski&  on& snow\\ \hline
         V&  \cellspace &  & \\ \hline 
         N&  \cellspace &  & \\ \hline 
         P&  \cellspace &  & \\ \hline 
    \end{tabular}
    \label{tab:my_label}
\end{table}
\subsection{Finding the final sequence}
What is the most probable POS sequence identified by the Viterbi algorithm?
\end{document}
