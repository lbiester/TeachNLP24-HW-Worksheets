\documentclass{article}
\usepackage{graphicx}
\usepackage[margin=1in]{geometry}
\usepackage{float}

% name of worksheet
\newcommand{\worksheetname}{Beam Search for Text Generation}


\begin{document}

{\Large\textbf{\worksheetname}}\\\rule{\linewidth}{0.5mm}
\pagenumbering{gobble}

\section{Beam Search}
In this problem, you will use the beam search algorithm to generate text given the prompt ``The sun is shining''. To get the probabilities, you should use the \texttt{next\_token\_probs} function in the beam search colab notebook that is linked on the course website.

\textbf{The first step is performed for you. Perform as many steps as you can. Use a beam width of 2, and turn the sheet sideways!}
\begin{figure}[H]
    \centering
\includegraphics[width=0.2\textwidth]{img/beam_search.png}
\end{figure}

\newpage
\section{Evaluating Generated Text with BLEU Score}
Compute the modified bigram precision given the following two reference texts:

\textbf{Reference 1:} The sun is shining on the horizon

\textbf{Reference 2:} The sun shines on the horizon



\begin{table}[H]
    \LARGE
    \centering
    \begin{tabular}{|l|p{1in}|p{1in}|} \hline 
         Candidate Text&  $p_1$& $p_2$\\ \hline 
         The shining sun rises over the horizon& & \\ \hline 
         Over the horizon is the sun&  & \\ \hline
    \end{tabular}
\end{table}

\textit{If you have extra time, you can compute precision scores for larger values of n.}

\end{document}
